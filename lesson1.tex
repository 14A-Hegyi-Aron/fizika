\documentclass{article}
\usepackage{graphicx}
\usepackage{amsmath}
\usepackage{mathtools}
\graphicspath{ {./images/} }

\title{Lesson 1}
\date{\today}
\author{Áron F. Hegyi,
\and GitHub Copilot}
\begin{document}
\maketitle

\section{Konjuktív és diszjunktív logikai kapcsolatok}

\subsection{Egyszerűbb írásmód: De Morgan's Laws}

\begin{table}[ht]
    \begin{tabular}{|c|c|c|c|}
        \hline
        \textbf{\(m_i^n\)}                      & \textbf{i} & \textbf{j} & \textbf{\(\overline{M_j^n}\)}                             \\
        \hline
        A * B * C                               & 7          & 0          & \(\overline{\overline{A} + \overline{B} + \overline{C}}\) \\
        \hline
        \(\overline{A}\) * B * \(\overline{C}\) & 2          & 5          & \(\overline{A + \overline{B} + C}\)                       \\
        \hline
        A * \(\overline{B}\) * C                & 5          & 2          & \(\overline{\overline{A} + B + \overline{C}}\)            \\
        \hline
    \end{tabular}
\end{table}

This is the sum of all the minterms of \(F^4\).
\begin{equation*}
    F^4 = \sum^{4} (0, 2, 3, 4, 5, 11, 15)
\end{equation*}

\(F^4\) = ABCD 7 times
\begin{multline*}
    F^4 = (\overline{A} + \overline{B} + \overline{C} + \overline{D}) \cdot (\overline{A} + \overline{B} + C + \overline{D}) \cdot (\overline{A} + \overline{B} + C + D) \cdot (\overline{A} + B + \overline{C} + \overline{D}) \\ \cdot (\overline{A} + B + \overline{C} + D) \cdot (\overline{A} + B + C + \overline{D}) \cdot (A + B + C + D)
\end{multline*}

\newpage
ABCD igazságtábla
\begin{table}[ht]
    \begin{tabular}{|c|c|c|c|c|c|}
        \hline
        \textbf{0} & \textbf{A} & \textbf{B} & \textbf{C} & \textbf{D} & \textbf{F} \\
        \hline
        1          & 0          & 0          & 0          & 0          & 0          \\
        \hline
        2          & 0          & 0          & 0          & 1          & 1          \\
        \hline
        3          & 0          & 0          & 1          & 0          & 1          \\
        \hline
        4          & 0          & 0          & 1          & 1          & 1          \\
        \hline
        5          & 0          & 1          & 0          & 0          & 1          \\
        \hline
        6          & 0          & 1          & 0          & 1          & 0          \\
        \hline
        7          & 0          & 1          & 1          & 0          & 0          \\
        \hline
        8          & 0          & 1          & 1          & 1          & 0          \\
        \hline
        9          & 1          & 0          & 0          & 0          & 0          \\
        \hline
        10         & 1          & 0          & 0          & 1          & 0          \\
        \hline
        11         & 1          & 0          & 1          & 0          & 1          \\
        \hline
        12         & 1          & 0          & 1          & 1          & 0          \\
        \hline
        13         & 1          & 1          & 0          & 0          & 0          \\
        \hline
        14         & 1          & 1          & 0          & 1          & 0          \\
        \hline
        15         & 1          & 1          & 1          & 0          & 1          \\
        \hline
    \end{tabular}
\end{table}

\section{Egyszerűsítés}
Logikai függvények egszerűsítése! Logikai algebra.
Szabályai és alkalmazásuk:

\subsection{Kommutatív szabály}
\begin{align*}
    A + B = B + A \\
    A \cdot B = B \cdot A
\end{align*}

\subsection{Disztributív szabály}
\begin{align*}
    A \cdot (B + C) = A \cdot B + A \cdot C \\
    A + (B \cdot C) = (A + B) \cdot (A + C)
\end{align*}

\subsection{A logikai algebra alapszabályai}
\[\begin{array}{lll}
        A \cdot \oslash = \oslash & A \cdot A = A & \oslash \cdot \oslash = \oslash \\
        A + \oslash = A & A + A = A & \oslash + \oslash = \oslash \\
        A \cdot 1 = A & A + 1 = 1 & 1 \cdot 1 = 1 \\
        A + 1 = 1 & A \cdot 1 = A & 1 + 1 = 1 \\
\end{array}\]

\subsection{De Morgan szabály}
\begin{align*}
    \overline{A \cdot B} = \overline{A} + \overline{B} \\
    \overline{A + B} = \overline{A} \cdot \overline{B}
\end{align*}
Igazságtábla:
\begin{table}[ht]
    \begin{tabular}{|c|c|c|c|}
        \hline
        \textbf{A} & \textbf{B} & \textbf{\(\overline{A \cdot B}\)} & \textbf{\(\overline{A} + \overline{B}\)} \\
        \hline
        0          & 0          & 1                             & 1                             \\
        \hline
        0          & 1          & 1                             & 1                             \\
        \hline
        1          & 0          & 1                             & 1                             \\
        \hline
        1          & 1          & 0                             & 0                             \\
        \hline
    \end{tabular}
\end{table}
\newline
De Morgan szabály megfordítja a logikai műveleteket a negációval.

\end{document}