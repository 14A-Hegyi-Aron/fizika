\documentclass{article}
\usepackage{graphicx}
\usepackage{amsmath}
\usepackage{mathtools}
\usepackage[a4paper, total={6in, 8in}, margin=1in]{geometry}
\graphicspath{ {./images/} }

\title{Lesson 1}
\date{\today}
\author{Áron F. Hegyi,
\and GitHub Copilot}
\begin{document}
\maketitle

\section{Konjuktív és diszjunktív logikai kapcsolatok}

\subsection{Egyszerűbb írásmód: De Morgan's Laws}

\begin{table}[ht]
    \begin{tabular}{|c|c|c|c|}
        \hline
        \textbf{\(m_i^n\)}                      & \textbf{i} & \textbf{j} & \textbf{\(\overline{M_j^n}\)}                             \\
        \hline
        A * B * C                               & 7          & 0          & \(\overline{\overline{A} + \overline{B} + \overline{C}}\) \\
        \hline
        \(\overline{A}\) * B * \(\overline{C}\) & 2          & 5          & \(\overline{A + \overline{B} + C}\)                       \\
        \hline
        A * \(\overline{B}\) * C                & 5          & 2          & \(\overline{\overline{A} + B + \overline{C}}\)            \\
        \hline
    \end{tabular}
\end{table}

This is the sum of all the minterms of \(F^4\).
\begin{equation*}
    F^4 = \sum^{4} (0, 2, 3, 4, 5, 11, 15)
\end{equation*}

\(F^4\) = ABCD 7 times
\begin{multline*}
    F^4 = (\overline{A} + \overline{B} + \overline{C} + \overline{D}) \cdot (\overline{A} + \overline{B} + C + \overline{D}) \cdot (\overline{A} + \overline{B} + C + D) \cdot (\overline{A} + B + \overline{C} + \overline{D}) \\ \cdot (\overline{A} + B + \overline{C} + D) \cdot (\overline{A} + B + C + \overline{D}) \cdot (A + B + C + D)
\end{multline*}

\newpage
ABCD igazságtábla
\begin{table}[ht]
    \begin{tabular}{|c|c|c|c|c|c|}
        \hline
        \textbf{0} & \textbf{A} & \textbf{B} & \textbf{C} & \textbf{D} & \textbf{F} \\
        \hline
        1          & 0          & 0          & 0          & 0          & 0          \\
        \hline
        2          & 0          & 0          & 0          & 1          & 1          \\
        \hline
        3          & 0          & 0          & 1          & 0          & 1          \\
        \hline
        4          & 0          & 0          & 1          & 1          & 1          \\
        \hline
        5          & 0          & 1          & 0          & 0          & 1          \\
        \hline
        6          & 0          & 1          & 0          & 1          & 0          \\
        \hline
        7          & 0          & 1          & 1          & 0          & 0          \\
        \hline
        8          & 0          & 1          & 1          & 1          & 0          \\
        \hline
        9          & 1          & 0          & 0          & 0          & 0          \\
        \hline
        10         & 1          & 0          & 0          & 1          & 0          \\
        \hline
        11         & 1          & 0          & 1          & 0          & 1          \\
        \hline
        12         & 1          & 0          & 1          & 1          & 0          \\
        \hline
        13         & 1          & 1          & 0          & 0          & 0          \\
        \hline
        14         & 1          & 1          & 0          & 1          & 0          \\
        \hline
        15         & 1          & 1          & 1          & 0          & 1          \\
        \hline
    \end{tabular}
\end{table}

\section{Egyszerűsítés}
Logikai függvények egszerűsítése! Logikai algebra.
Szabályai és alkalmazásuk:

\subsection{Kommutatív szabály}
\begin{align*}
    A + B = B + A \\
    A \cdot B = B \cdot A
\end{align*}

\subsection{Disztributív szabály}
\begin{align*}
    A \cdot (B + C) = A \cdot B + A \cdot C \\
    A + (B \cdot C) = (A + B) \cdot (A + C)
\end{align*}

\subsection{A logikai algebra alapszabályai}
\[\begin{array}{lll}
        A \cdot \oslash = \oslash & A \cdot A = A & \oslash \cdot \oslash = \oslash \\
        A + \oslash = A           & A + A = A     & \oslash + \oslash = \oslash     \\
        A \cdot 1 = A             & A + 1 = 1     & 1 \cdot 1 = 1                   \\
        A + 1 = 1                 & A \cdot 1 = A & 1 + 1 = 1                       \\
    \end{array}\]

\subsection{De Morgan szabály}
\begin{align*}
    \overline{A \cdot B} = \overline{A} + \overline{B} \\
    \overline{A + B} = \overline{A} \cdot \overline{B}
\end{align*}
Igazságtábla:
\begin{table}[ht]
    \begin{tabular}{|c|c|c|c|}
        \hline
        \textbf{A} & \textbf{B} & \textbf{\(\overline{A \cdot B}\)} & \textbf{\(\overline{A} + \overline{B}\)} \\
        \hline
        0          & 0          & 1                                 & 1                                        \\
        \hline
        0          & 1          & 1                                 & 1                                        \\
        \hline
        1          & 0          & 1                                 & 1                                        \\
        \hline
        1          & 1          & 0                                 & 0                                        \\
        \hline
    \end{tabular}
\end{table}
\newline
De Morgan szabály megfordítja a logikai műveleteket a negációval.

\subsection{XOR és XNOR}
\begin{align*}
    A \oplus B & = \overline{A} \cdot B + A \cdot \overline{B} \\
    A \odot B  & = \overline{A} \cdot \overline{B} + A \cdot B
\end{align*}

\subsection{Gyakorlás}
\begin{align*}
    A \oplus B                                  & = \overline{A \odot B}                                 \\
    \overline{A} \cdot B + A \cdot \overline{B} & = \overline{A} \cdot \overline{B} + A \cdot B          \\
    \overline{A} \cdot B + A \cdot \overline{B} & =
    (A + B) \cdot (\overline{A} + \overline{B})                                                          \\
    \overline{A} \cdot B + A \cdot \overline{B} & =
    A \cdot \overline{A} + A \cdot \overline{B} + \overline{B} \cdot \overline{A} + B \cdot \overline{B} \\
    \overline{A} \cdot B + A \cdot \overline{B} & =
    A \cdot \overline{B} + B \cdot \overline{A}                                                          \\
\end{align*}

\subsection{Logikai függvények egyszerűsítése grafikus módszerrel}
A
\begin{table}[ht]
    \begin{tabular}{c|c|}
        0 & \(\overline{A}\) \\
        \hline
        1 & \(A\)            \\
    \end{tabular}
\end{table}
\newline
A B
\begin{table}[ht]
    \begin{tabular}{c|c|c|}
          & 0                                   & 1                        \\
        \hline
        0 & \(\overline{A} \cdot \overline{B}\) & \(\overline{A} \cdot B\) \\
        \hline
        1 & \(A \cdot \overline{B}\)            & \(A \cdot B\)            \\
    \end{tabular}
\end{table}
\newline
A BC
\begin{table}[ht]
    \begin{tabular}{c|c|c|c|c|}
          & 00 & 01 & 11 & 10 \\
        \hline
        0 & 0  & 1  & 3  & 2  \\
        \hline
        1 & 4  & 5  & 7  & 6  \\
    \end{tabular}
\end{table}
\newpage
AB CD
\begin{table}[ht]
    \begin{tabular}{c|c|c|c|c|}
           & 00 & 01 & 11 & 10 \\
        \hline
        00 & 0  & 1  & 3  & 2  \\
        \hline
        01 & 4  & 5  & 7  & 6  \\
        \hline
        11 & 12 & 13 & 15 & 14 \\
        \hline
        10 & 8  & 9  & 11 & 10 \\
    \end{tabular}
\end{table}

\begin{align*}
    F^3 & = A \cdot B \cdot C + \overline{A} \cdot B \cdot C + A \cdot \overline{B} \cdot C + A \cdot B \cdot \overline{C} + \overline{A} \cdot \overline{B} \cdot C \\
\end{align*}
\begin{align*}
    \overline{A} \cdot \overline{B} \cdot \overline{C} + \overline{A} \cdot B \cdot C + A \cdot \overline{B} \cdot \overline{C} + A \cdot B \cdot \overline{C} + A \cdot B \cdot C \\
    \overline{A} \cdot \overline{B} \cdot \overline{C} + A \cdot \overline{C} + B \cdot C                                                                                          \\
    \overline{C} (\overline{A} \cdot A) + B \cdot C & =
    \overline{C} \cdot (\overline{B} + A) + B \cdot C                                                                                                                              \\
\end{align*}

\subsection{Karnaugh táblák}
A B
\begin{table}[ht]
    \begin{tabular}{c|c|c||c}
          & \textbf{A} & \textbf{B} & \textbf{\(F^2\)} \\
        \hline
        0 & 0          & 0          & 0                \\
        \hline
        1 & 0          & 1          & 1                \\
        \hline
        2 & 1          & 0          & 1                \\
        \hline
        3 & 1          & 1          & 0                \\
    \end{tabular}
\end{table}

A B C
\begin{table}[ht]
    \begin{tabular}{c|c|c|c||c}
          & \textbf{A} & \textbf{B} & \textbf{C} & \textbf{\(F^3\)} \\
        \hline
        0 & 0          & 0          & 0          & 1                \\
        \hline
        1 & 0          & 0          & 1          & 0                \\
        \hline
        2 & 0          & 1          & 0          & 1                \\
        \hline
        3 & 0          & 1          & 1          & 1                \\
        \hline
        4 & 1          & 0          & 0          & 0                \\
        \hline
        5 & 1          & 0          & 1          & 0                \\
        \hline
        6 & 1          & 1          & 0          & 1                \\
        \hline
        7 & 1          & 1          & 1          & 0                \\
    \end{tabular}
\end{table}

\begin{align*}
    \overline{A} \cdot \overline{B} \cdot \overline{C} + \overline{A} \cdot B \cdot \overline{C} + \overline{A} \cdot B \cdot C + A \cdot B \cdot \overline{C} + A \cdot B \cdot C \\
    \overline{A} \cdot \overline{C} \cdot (\overline{B} + B) + \overline{A} \cdot B \cdot C + A \cdot B \cdot (\overline{C} + C)                                                   \\
    \overline{A} \cdot \overline{C} + \overline{A} \cdot B \cdot C + A \cdot B                                                                                                     \\
    \overline{A} \cdot \overline{C} + B \cdot (\overline{A} \cdot C + A) & = \overline{A} \cdot \overline{C} + B \cdot (A + C)                                                     \\
\end{align*}

\begin{align*}
    F^3 & = \overline{A} \cdot \overline{C} + B \\
\end{align*}

\newpage
A B C D
\begin{table}[ht]
    \begin{tabular}{c|c|c|c|c||c}
           & \textbf{A} & \textbf{B} & \textbf{C} & \textbf{D} & \textbf{\(F^4\)} \\
        \hline
        0  & 0          & 0          & 0          & 0          & 1                \\
        \hline
        1  & 0          & 0          & 0          & 1          & 1                \\
        \hline
        2  & 0          & 0          & 1          & 0          & 1                \\
        \hline
        3  & 0          & 0          & 1          & 1          & 0                \\
        \hline
        4  & 0          & 1          & 0          & 0          & 0                \\
        \hline
        5  & 0          & 1          & 0          & 1          & 1                \\
        \hline
        6  & 0          & 1          & 1          & 0          & 1                \\
        \hline
        7  & 0          & 1          & 1          & 1          & 0                \\
        \hline
        8  & 1          & 0          & 0          & 0          & 1                \\
        \hline
        9  & 1          & 0          & 0          & 1          & 1                \\
        \hline
        10 & 1          & 0          & 1          & 0          & 1                \\
        \hline
        11 & 1          & 0          & 1          & 1          & 0                \\
        \hline
        12 & 1          & 1          & 0          & 0          & 0                \\
        \hline
        13 & 1          & 1          & 0          & 1          & 1                \\
        \hline
        14 & 1          & 1          & 1          & 0          & 1                \\
        \hline
        15 & 1          & 1          & 1          & 1          & 1                \\
    \end{tabular}
\end{table}

\begin{align*}
    \overline{B} \cdot \overline{C} + \overline{C} \cdot D + C \cdot \overline{D} + A \cdot D \cdot B \\
\end{align*}

\subsection{Feladat}

\begin{equation*}
    F^3 = \sum^{3} (1, 3, 4, 6)
\end{equation*}

ABC 3 times: 0001, 0011, 0100, 0110

\begin{align*}
    F^3 & = \overline{A} \cdot \overline{B} \cdot C + \overline{A} \cdot B \cdot C + A \cdot \overline{B} \cdot C + A \cdot B \cdot C \\
\end{align*}

Karnaugh tábla
A BC

\begin{table}[ht]
    \begin{tabular}{c|c|c|c|c}
          & 00 & 01 & 11 & 10 \\
        \hline
        0 & 0  & 1  & 1  & 0  \\
        \hline
        1 & 1  & 0  & 0  & 1  \\
    \end{tabular}
\end{table}

\begin{align*}
    F^3 & = A \cdot \overline{C} + \overline{A} \cdot C \\
\end{align*}

\begin{figure}[ht]
    \centering
    \includegraphics[width=0.7\textwidth]{1-es feladat ábra.png}
    \caption{feladat ábra}
    \label{fig:feladat ábra}
\end{figure}

https://www.falstad.com/circuit/circuitjs.html?ctz=CQAgjCAMB0l3BWcMBMcUHYMGZIA4UA2ATmIxAUgpABZsKBTAWjDACgB3EbPGkFFHzop+gqJ1F9s2QuEJVpsyGzCVJ3NN14aqVEUgQq1Avia2mxe6odUjh-DPuIjMIqwYn2zXhEok8pTXtsTWUuYKCQ-jw8cS4zNFiweX58cQBJOT0aPTFBXShoD0zkhUgLKXKoQo8AGSzU2OJZROqIADMAQwAbAGcGamV60v4ckGbRgqouvoGkIYbcPgml6ume-sG2YZTFcdk9qg6NufEAWQpnBxFCX2vqlCK2IA

\end{document}