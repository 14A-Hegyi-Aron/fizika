\documentclass{article}
\usepackage{graphicx}
\usepackage{amsmath}
\graphicspath{ {./images/} }
\newcommand*{\contfrac}[2]{%
{
  \rlap{$\dfrac{1}{\phantom{#1}}$}%
  \genfrac{}{}{0pt}{0}{}{#1+#2}%
}}

\title{The Ultimate \LaTeX\ Guide}
\date{\today}
\author{Áron F. Hegyi,
\and GitHub Copilot}
\begin{document}
\maketitle
\section{Introduction}

\subsection{Basic Commands}
\begin{itemize}
    \item \textbf{Bold} \textit{Italic} \underline{Underline}
\end{itemize}

\subsection{Math}
\begin{itemize}
    \item \(\frac{1}{2}\) \(\sqrt{2}\) \(x^2\) \(x_{i}\)
    \item \(x_{i} = \frac{1}{2}\)
\end{itemize}

\subsection{Tables}
\begin{table}[ht]
    \begin{tabular}{|c|c|c|}
        \hline
        \textbf{Column 1} & \textbf{Column 2} & \textbf{Column 3} \\
        \hline
        1                 & 2                 & 3                 \\
        \hline
        4                 & 5                 & 6                 \\
        \hline
    \end{tabular}
\end{table}

% \newpage

\subsection{Figures}
\begin{figure}[ht]
    \centering
    \includegraphics[width=0.25\textwidth]{test.png}
    \caption{This is a scary figure. I advise running.}
\end{figure}

\section{Mathematics}

\subsection{Subscripts and Superscripts}
\begin{itemize}
    \item Create a subscript with the \_ character: \(x_{i}\)
    \item Create a superscript with the \textsuperscript{$\wedge$} character: \(x^{2}\)
    \item You can also combine them: \(x_{i}^{2}\)
\end{itemize}

\subsection{Brackets and Parentheses}
\begin{itemize}
    \item Round brackets: (function arguments) \[\left( \frac{1}{2} \right)\]
    \item Square brackets: (vectors) \[\left[ \frac{1}{2} \right]\]
    \item Curly brackets: (sets) \[\left\{ \frac{1}{2} \right\}\]
    \item Absolute value: (absolute values) \[\left| \frac{1}{2} \right|\]
    \item Floor: (floor function, i.e. rounding down) \[\left\lfloor \frac{1}{2} \right\rfloor\]
    \item Ceiling (ceiling function, i.e. rounding up) \[\left\lceil \frac{1}{2} \right\rceil\]
    \item Angle brackets: (inner product) \[\left\langle \frac{1}{2} \right\rangle\]
    \item Double angle brackets: (outer product) \[\left\langle \left\langle \frac{1}{2} \right\rangle \right\rangle\]
\end{itemize}

\subsection{Matrices}
\begin{itemize}
    \item Plain: a matrix with no brackets \[\begin{matrix} 1 & 2 \\ 3 & 4 \end{matrix}\]
    \item Parentheses: a matrix with round brackets \[\begin{pmatrix} 1 & 2 \\ 3 & 4 \end{pmatrix}\]
    \item Square brackets: a matrix with square brackets \[\begin{bmatrix} 1 & 2 \\ 3 & 4 \end{bmatrix}\]
    \item Curly brackets: a matrix with curly brackets \[\begin{Bmatrix} 1 & 2 \\ 3 & 4 \end{Bmatrix}\]
    \item Absolute value: a matrix with absolute value brackets \[\begin{vmatrix} 1 & 2 \\ 3 & 4 \end{vmatrix}\]
    \item Double absolute value: a matrix with double absolute value brackets \[\begin{Vmatrix} 1 & 2 \\ 3 & 4 \end{Vmatrix}\]
\end{itemize}


\subsection{Fractions and Binomials}
\begin{itemize}
    \item Binomials: \[\binom{n}{k}\]
    \item Fractions with numbers: \[\frac{1}{2}\]
    \item Fractions with text: \[\frac{numerator}{denominator}\]
    \item Function arguments: \[f(x)=\frac{f(x+h)-f(x)}{h}\]
    \item Nested fractions: \[\frac{1}{1+\frac{1}{x}}\]
    \item Fractions with subscripts: \[\frac{x_{i}}{x_{i+1}}\]
    \item Overflow example: \[
              a_0 +
              \contfrac{a_1}{
                  \contfrac{a_2}{
                      \contfrac{a_3}{
                          \genfrac{}{}{0pt}{0}{}{\ddots}
                      }}}
          \]
\end{itemize}

\subsection{Aligning Equations}
\begin{itemize}
    \item Aligning equations with the \texttt{align} environment: \begin{align}
              a & = b \\
              c & = d
          \end{align}
    \item Use the asterisk to align equations without numbers: \begin{align*}
              a & = b \\
              c & = d
          \end{align*}
    \item Two columns: \begin{align*}
              a & = b & c & = d \\
              e & = f & g & = h
          \end{align*}
    \item Lopsided: \begin{align*}
              \frac{1}{2} & = a \\
                          & = c
          \end{align*}
    \item Displaying large equations: \begin{multline*}
              p(x) = 3x^6 + 14x^5y + 590x^4y^2 + 19x^3y^3 \\
              + 14x^2y^4 + 3xy^5 + y^6 + 3x^5 + 14x^4y + 19x^3y^2
          \end{multline*}
    \item Grouping equations: \begin{gather*}
              a = b \\
              c = a + b
          \end{gather*}
\end{itemize}

\subsection{Operators}
\begin{itemize}
    \item Summation: to calculate the sum of a series
          \[\sum_{i=1}^{n} x_{i}\] the sum of \(x_{i}\) from \(i=1\) to \(i=n\)
    \item Product: to calculate the product of a series
          \[\prod_{i=1}^{n} x_{i}\] the product of \(x_{i}\) from \(i=1\) to \(i=n\)
    \item Union: \[\bigcup_{i=1}^{n} x_{i}\]
    \item Intersection: \[\bigcap_{i=1}^{n} x_{i}\]
    \item Integral: \[\int_{a}^{b} x\]
    \item Double integral: \[\iint_{a}^{b} x\]
    \item Triple integral: \[\iiint_{a}^{b} x\]
    \item Contour integral: \[\oint_{a}^{b} x\]
    \item Limit: \[\lim_{x \to 0} x\]
    \item Logarithm: \[\log_{2} x\]
    \item Derivative: \[\frac{d}{dx} x\]
    \item Partial derivative: \[\frac{\partial}{\partial x} x\]
\end{itemize}


\subsubsection{The Simpler Quadratic Formula}
\begin{align*}
    m \pm \sqrt{m^2 - c} &= \frac{-b \pm \sqrt{b^2 - 4ac}}{2a} \\
    m = -\frac{b}{2a} \\
    b = - (x + y) \\
    c = x \cdot y \\
    \\
    x^2+4x+3 &= 0 \\
    m &= -\frac{b}{2} = -2 \\
    d &= \sqrt{m^2 - c} = \sqrt{4 - 3} = \sqrt{1} \\
    x,y &= m \pm d = -2 \pm \sqrt{1} = -1, -3 \\
    \\
    old \\
    x^2+4x+3 &= 0 \\
    x &= \frac{-b \pm \sqrt{b^2 - 4ac}}{2a} \\
    x &= \frac{-4 \pm \sqrt{16 - 4 \cdot 1 \cdot 3}}{2} \\
    x &= \frac{-4 \pm \sqrt{4}}{2} \\
    x &= \frac{-4 \pm 2}{2} \\
    x &= -1, -3 \\
    \\
    x^2-10x+6 &= 0 \\
    m &= -\frac{b}{2} = 5 \\
    d &= \sqrt{m^2 - c} = \sqrt{25 - 6} = \sqrt{19} \\
    x,y &= m \pm d = 5 \pm \sqrt{19} = 5 \pm 4.3589 = 9.3589, 0.6411 \\
    \\
    x^2-6x+10 &= 0 \\
    m &= -\frac{b}{2} = 3 \\
    d &= \sqrt{m^2 - c} = \sqrt{9 - 10} = \sqrt{-1} \\
    x,y &= m \pm d = 3 \pm \sqrt{-1} = 3 \pm i \\
    \\
    3x^2-5x+2 &= 0 \\
    x^2 - \frac{5}{3}x + \frac{2}{3} &= 0 \\
    m &= -\frac{b}{2} = \frac{5}{6} \\
    d &= \sqrt{m^2 - c} = \sqrt{\frac{25}{36} - \frac{2}{3}} = \sqrt{\frac{1}{36}} = \frac{1}{6} \\
    x,y &= m \pm d = \frac{5}{6} \pm \frac{1}{6} = 1, \frac{2}{3} \\
\end{align*}


\end{document}
